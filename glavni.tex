\documentclass[a4paper,12pt]{report}
\usepackage[utf8]{inputenc} %s ovim mogu pisati hrv znakove direktno na linux i win. platformama (ali samo u utf-8 encodingu)
\usepackage{amsmath}
%\usepackage[english,croatian]{babel}
\usepackage[croatian]{babel}
\usepackage{xspace}
\usepackage{url}
\usepackage{hangcaption}
\usepackage[croatian,refpage]{nomencl}
\makenomenclature
%\usepackage[master,avant]{fsbthesis}
\usepackage[zavrsni,avant]{fsbthesis}
    % ili npr. \usepackage[avant,draft]{fsbthesis} %za doktorat - draft
	%   ispisuje broj verzije teksta (promijeni naredbu \ver) i datum
    % ili npr. \usepackage[zavrsni,avant]{fsbthesis} %za zavrsni rad
    % ili npr. \usepackage[diplomski,avant]{fsbthesis} %za diplomski rad
	% ili npr. \usepackage[diplomski,avant,draft]{fsbthesis} % za diplomski rad uz draft opciju - dodaje oznake datuma/ver.
    % ili npr. \usepackage[klasik]{fsbthesis} %za klasican izgled doktorata
	% ili npr. \usepackage[master,avant]{fsbthesis} za dipl.rad na engl. ondnosno master's thesis
	% ... za opis svih opicija pogledaj fsbthesis.sty (nazalost nije osvjezeno zadnjim promjenama)
	% za vizualni izgled rada na raspolaganju su tri opcije: avant, alter i klasik
%\usepackage{t1enc}
\usepackage{tikz}
\usetikzlibrary{arrows,calc}

\vfuzz=2.7pt % Don't bother to report overfull boxes < 2.7pt
%\graphicspath{{slike/}} % dodaj poseban direktoriji za slike (vrlo prakticno)

%-- sljedece elemente je potrebno prilagoditi vasem radu:
\title{NASLOV DIPLOMSKOG RADA}  %-- promijeniti naziv rada
\author{ime prezime}      %-- promijeniti autora rada
%\date{Zagreb, May 2003}      % -- promijeniti datum: mjesec i godina - EN (za master opciju)
\date{Zagreb, svibanj 2003.}      % -- promijeniti datum: mjesec i godina - HR

\makeatletter
\renewcommand{\FSBautor}{Moje ime i prezime} %-- promijeniti ime autora
\renewcommand{\FSBmentor}{prof.~dr.~sc.~ime prezime} %-- promijenti titulu i ime mentora 
\renewcommand{\FSBgodina}{2003.} %-- promijeniti godinu obrane rada
\renewcommand{\FSBkljucnerijeci}{popis klju\v{c}nih rije\v{c}i: maksimalno do deset \ldots} %-- promijeniti popis kljucnih rijeci
\renewcommand{\FSBkeywords}{list of the keywords: up to ten words} %-- promijeniti popis kljucnih rijeci na engleskom
%-- sljedeci elementi potrebni su za doktorsku disertaciju:
%\renewcommand{\FSBbibmentor}{
%		dr.~sc.~ime prezime,
%		prof. u mirovini}
%\renewcommand{\FSBbibudk}{xxx.xx}
%\renewcommand{\FSBbibznpolje}{zrakoplovstvo, raketna i svemirska tehnika}
%\renewcommand{\FSBbibdatumobr}{1. travnja 2005.}
% \renewcommand{\FSBbibznpolje}{zrakoplovstvo, raketna i svemirska tehnika}
% \renewcommand{\FSBbibdatumobr}{1. travnja 2005.}
% \renewcommand{\FSBbibpovjerenstvo}{prvi \v{c}lan povjerenstva, \newline
%                                    drugi \v{c}lan povjerenstva, \newline
%                                    tre\'{c}i \v{c}lan povjerenstva, \newline
%                                    \v{c}etvrti \v{c}lan povjerenstva, \newline
%                                    peti \v{c}lan povjerenstva }
% \renewcommand{\FSBpdfbrojstrana}{111}
% \renewcommand{\FSBpdfbrojslika}{111}
% \renewcommand{\FSBpdfbrojtablica}{111}
% \renewcommand{\FSBpdfbrojrefs}{111}
\renewcommand{\thestrana}{11}

\renewcommand{\ver}{\mbox{ver.2-01}} % moja verzija ovog rada (od interesa uz draft opciju)

\makeatother

% osobne definicije
\def\AR{\hbox{\textit{A}\hskip-.38em \rm R}} % aspect ratio sign
\def\Vinf{\hbox{$V_{\infty}$}} %brzina strujanja


\begin{document}
%
\pagenumbering{roman} % na pocetku rada stranice brojim rimski od (i)
%
\fsbkorice %-- naredba koja generira korice rada
%
\fsbprvilist %-- naredba koja generira prvi list iza korica
%
%\setcounter{brojstrana}{11}
%\fsbbiblist
%
%-- unijeti svoju zahvalu
\begin{zahvala}
	\emph{To Donald E. Knuth \dots i druge zahvale \dots}
\end{zahvala}

%-- nije potrebno mijenjati
\begin{izjava}
    %-- dio izjave na EN (suvišan za dipl.rad na HR)
    % I hereby declare that I have made this thesis independently using the
	% knowledge acquired during my studies and the cited references.
	% 
	% \bigskip
	% 
    %-- dio izjave na HR
	Izjavljujem da sam ovaj rad radio samostalno koristeći znanja stečena
	tijekom studija i navedenu literaturu.
\end{izjava}

%-- ako je potrebno ovdje se moze dodati predgovor (ali to je vise primjereno dokt.disertacijama)
% \begin{predgovor}
% Ovdje \'{c}e se opisati \v{s}to me i za\v{s}to navelo na ovaj rad. Mogu\'{c}e je i
% ovdje staviti zahvale.
% 
% Tekst tekst tekst tekst tekst tekst.
% \end{predgovor}

%-- sadrzaj:
\clearpage
\ifpdf
	\phantomsection
\fi
\addcontentsline{toc}{chapter}{{\numberline{}\contentsname}}
\tableofcontents

%
%-- popis slika: ukoliko ih ima
\clearpage
\ifpdf
	\phantomsection
\fi
\addcontentsline{toc}{chapter}{{\numberline{}\listfigurename}}
\listoffigures

%-- popis tablica: ukoliko ih ima
\clearpage
\addcontentsline{toc}{chapter}{{\numberline{}\listtablename}}
\listoftables

%-- ako se koristi klasicni popis oznaka
%\addcontentsline{toc}{chapter}{{\numberline {}\popoznime}}
%	\markboth{\MakeUppercase\popoznime}{}
%\input popozn.tex

\clearpage
\ifpdf
	\phantomsection
\fi
\addcontentsline{toc}{chapter}{{\numberline {}\popoznime}}
\markboth{\MakeUppercase\nomname}{\MakeUppercase\nomname}
\printnomenclature

\clearpage
\addtocounter{brojstrana}{\arabic{page}}\addtocounter{brojstrana}{-1}
    % nuzna komanda za odredjivanje ukupnog broja str.(rimskih i arapskih)

%-- promijeniti sazetak
\begin{sazetak}
  \noindent Kratki sa\v{z}etak rada na hrvatskom jeziku: najvi\v{s}e jedna stranica
  \ldots Tekst tekst tekst tekst tekst tekst.
\end{sazetak}

%-- promijenti sazetak na EN
\begin{abstract}
  \noindent Short summary of the thesis in one foreign language (english): up
  to one page \ldots Tekst tekst tekst tekst tekst tekst.
\end{abstract}


%-- kada je tekst na engleskom potrebno je dodati prošireni sažetak na hrvatskom:
%   (za HR diplomski rad ovo je suvisno)
% \begin{prosirenisazetak}
% Za rad pisan na engleskom jeziku potrebno je napisati ``Prošireni sažetak'' koji
% treba sadržavati više najbitnijih detalja i zaključaka diplomskog rada. Tekst
% nije ograničen, ali poželjno je da ima nekoliko stranica. 
% 
% Tekst tekst tekst tekst tekst tekst.
% Tekst tekst tekst tekst tekst tekst.
% Tekst tekst tekst tekst tekst tekst.
% Tekst tekst tekst tekst tekst tekst.
%  
% \section{Tekst}
% U ovom proširenom sažetku moguće je strukturirati tekst u poglavlja i
% podpoglavlja. Pri tome, ako je zahtjev da poglavlja budu numerirana onda će
% broj poglavlja imati prefiks ``0''. Tako će se razlikovati od glavnog dijela teksta. 
% Ova poglavlja i podpoglavlja pojavit će se i u ``Sadržaju'' odnosno u
% ``Contents''.
% 
% Tekst tekst tekst tekst tekst tekst.
% Tekst tekst tekst tekst tekst tekst.
% Tekst tekst tekst tekst tekst tekst.
% Tekst tekst tekst tekst tekst tekst.
%  
% 
% \subsection{Drugi tekst}
% Tekst tekst tekst tekst tekst tekst.
% Tekst tekst tekst tekst tekst tekst.
% Tekst tekst tekst tekst tekst tekst.
% Tekst tekst tekst tekst tekst tekst.
%  
% 
% \section*{Tekst}
% Ukoliko numeracija poglavlja i podpoglavlja nije od interesa dovoljno ih je u
% definiciji označiti sa ``*'', odnosno komanda za poglavlje bi bila
% \verb|\section*{Tekst}|. Slično se modificira i naredba \verb|\subsection*|,
% itd. Ova poglavlja i podpoglavlja neće se pojavit u ``Sadržaju'' odnosno u
% ``Contents''.
% 
% Tekst tekst tekst tekst tekst tekst.
% Tekst tekst tekst tekst tekst tekst.
% Tekst tekst tekst tekst tekst tekst.
% Tekst tekst tekst tekst tekst tekst.
%  
% 
% \subsection*{Drugi tekst}
% I ``Proširenom sažetku'' moguće je navesti (odnosno ponoviti) i najbitnije slike, slike s
% najvažnijim rezultatima i sl. (kao npr. Sliku~\ref{fig:ps.1}).
% Slike u ``Proširenom sažetku'' bit će numerirane s prefiksom ``0'' i pojavit
% će se u ``List of Figures''. Na isti način uvode se i tablice.
% \begin{figure}
%   \centering
%   \includegraphics[height=3.2cm]{unizg_sivi_t4s}\\
%   \hangcaption{Primjer slike u ``Proširenom sažetku''; kod ove slike primjenjeno je
%   \emph{vise\'{c}e} zaglavlje -- hangcaption}
%   \label{fig:ps.1}
% \end{figure}
% 
% \begin{figure}
%   \centering
%   \includegraphics[height=3.2cm]{unizg_plavi_t2}\\
%   \hangcaption{Primjer slike u ``Proširenom sažetku''; kod ove slike primjenjeno je
%   \emph{vise\'{c}e} zaglavlje -- hangcaption}
% \end{figure}
% 
% \begin{figure}
%   \centering
%   \includegraphics[width=3.2cm]{fsb_logo_n}\\
%   \hangcaption{Primjer slike u ``Proširenom sažetku''; kod ove slike primjenjeno je
%   \emph{vise\'{c}e} zaglavlje -- hangcaption}
% \end{figure}
% 
% \begin{figure}
%   \centering
%   \includegraphics[height=2.6cm]{fsb_logo_v}\\
%   \hangcaption{Primjer slike u ``Proširenom sažetku''; kod ove slike primjenjeno je
%   \emph{vise\'{c}e} zaglavlje -- hangcaption}
% \end{figure}
% 
% Tekst tekst tekst tekst tekst tekst.
% Tekst tekst tekst tekst tekst tekst.
% Tekst tekst tekst tekst tekst tekst.
% Tekst tekst tekst tekst tekst tekst.
% 
% U ``Proširenom sažetku'' također je moguće unijeti najbitnije jednadžbe iz
% rada poput~\eqref{eq:ps.1}
% \begin{equation}\label{eq:ps.1}
% 	C_l(y)=\frac{2\Gamma(y)}{\Vinf\;c(y)}\:.
% \end{equation}
% Ukoliko numeracija jednadžbi nije od interesa ista se može ispustiti
% \[
% \Delta C_p=\frac{2}{\Vinf}\cdot\gamma\:.
% \]
% 
% 
% Tekst tekst tekst teks tekst tekst.
% Tekst tekst tekst teks tekst tekst.
% Tekst tekst tekst teks tekst tekst.
% Tekst tekst tekst teks tekst tekst.
% \end{prosirenisazetak}


%-- ovdje pocinje pravi tekst: najelegantnije jest ubaciti ga kao posebne .tex
%datoteke - jednu za cijeli tekst ili po jednu za svako poglavlje ili ...
\clearpage
\pagenumbering{arabic} % mjenjam brojke stranice: ali od 1
\input uvod.tex
\addtocounter{brojslika}{\arabic{figure}}
\addtocounter{brojtablica}{\arabic{table}}

\input teorija.tex
\addtocounter{brojslika}{\arabic{figure}}
\addtocounter{brojtablica}{\arabic{table}}

\input rezultati.tex
\addtocounter{brojslika}{\arabic{figure}}
\addtocounter{brojtablica}{\arabic{table}}
% ... i tako dalje za ostala poglavlja

\input zakljuc.tex
\addtocounter{brojslika}{\arabic{figure}}
\addtocounter{brojtablica}{\arabic{table}}

\appendix
\chapter{Moj prvi dodatak}
\addtocounter{brojslika}{\arabic{figure}}
\addtocounter{brojtablica}{\arabic{table}}
Tu dolazi prvi prilog odnosno dodatak tekstu (slike, podaci, kod, detaljni
opisi, tablice, \ldots
\newpage 
Pa ni na ovoj stranici!
\section{Malo poglavlje malog dodatka}
\subsection{i još manje podpoglavlje}
Ovo poglavlje poslu\v{z}it \'{c}e za uvod u problem koji se rje\v{s}ava u ovom
radu, te se postavlja odgovaraju\'{c}a hipoteza za doktorsku
disertaciju.

\section{Primjer potpoglavlja}
Tekst tekst tekst tekst tekst tekst primjer reference
\cite{mastersthesis-minimal}, i jo\v{s} jedan \cite{inbook-full}.
Tekst tekst tekst tekst tekst tekst 

Malo ću se pozvati (\eqref{eq:ls-a1}) kao u šđč枊ĐČĆŽ na 
\begin{equation}\label{eq:ls-a1}
	C_l(y)=\frac{2\Gamma(y)}{\Vinf\;c(y)}\:,
\end{equation}
\nomenclature[acal]{$C_l$}{lokalni koeficijent uzgona}%\refeq}%
\nomenclature[ecax]{$c$}{duljina tetive, [m]}%

\subsection{Primjer potpotpoglavlja}

Slijedi prvi primjer slike: FSB.(pogl.sliku~\ref{figa1})
\nomenclature[Kf]{$FSB$}{Fakultet strojarstva i brodogradnje}%

\begin{figure}[h]
  \centering
  \includegraphics[height=1.2cm]{fsb_logo_n}\\
  \hangcaption{Primjer slike: kod ove slike primjenjeno je
  \emph{vise\'{c}e} zaglavlje -- hangcaption}
  \label{figa1}
\end{figure}

\clearpage
Slijedi mali primjer slike: SUZ.(pogledaj sliku~\ref{figa2})
\nomenclature[Ks]{$SUZ$}{Sveučilište u Zagrebu}%
%prvim slovom unutar uglate zagrade 'K' definiram da li je kratica ili indeks ili
%akcent, a drugim slovom 's' mogu regulirati poredak pojavljivanja u popisu
%oznaka (po abecedi, naravno)

\begin{figure}
  \centering
  \includegraphics[height=3.2cm]{unizg_sivi_t4s}\\
  \hangcaption{Primjer slike: kod ove slike primjenjeno je
  \emph{vise\'{c}e} zaglavlje -- hangcaption}
  \label{figa2}
\end{figure}


Pored slike dan je i primjer tablice (\ref{tablicaa}).

\begin{table}[!h]
\hangcaption{Primjer tablice}
\label{tablicaa}
  \centering
\begin{tabular}{|c c|} \hline
    A & 1 \\
    B & 2 \\
    C & 3 \\
    D & 4 \\ \hline
\end{tabular}
\end{table}

\chapter{i drugi \dots}
\addtocounter{brojslika}{\arabic{figure}}
\addtocounter{brojtablica}{\arabic{table}}
\dots Opet ništa osim ove slike:~\ref{dodatna}
\begin{figure}
  \centering
  \includegraphics[height=3.2cm]{fsb_logo_v}
  \hangcaption{Primjer slike u prilogu}
  \label{dodatna}
\end{figure}
\newpage
\section{Samo prašim po testu}
Ovo poglavlje poslu\v{z}it \'{c}e za uvod u problem koji se rje\v{s}ava u ovom
radu, te se postavlja odgovaraju\'{c}a hipoteza za doktorsku
disertaciju.


\nomenclature[Kf]{$dod$}{dummy dodatak}%

% literatura:
\newpage
\ifpdf
	\phantomsection
\fi
\bibliographystyle{unsrt} %abbrv
\addcontentsline{toc}{chapter}{{\numberline{}\bibname}}
\bibliography{xampl}

\end{document}               % End of document.

%-- naredba kojom se generira popis oznaka
% makeindex glavni.nlo -s nomencl.ist -o glavni.nls 


