\chapter{Uvod}

Ovo poglavlje poslu\v{z}it \'{c}e za uvod u problem koji se rje\v{s}ava u ovom
radu, te se postavlja odgovaraju\'{c}a hipoteza za doktorsku
disertaciju.

\section{Primjer potpoglavlja}
Tekst tekst tekst tekst tekst tekst primjer reference
\cite{mastersthesis-minimal}, i jo\v{s} jedan \cite{inbook-full}.
Tekst tekst tekst tekst tekst tekst 

Malo żu se pozvati kao u (šđćčž-ŠĐČĆŽ) na
\begin{equation}\label{eq:ls-1}
	C_l(y)=\frac{2\Gamma(y)}{\Vinf\;c(y)}\:,
\end{equation}
\nomenclature[acal]{$C_l$}{lokalni koeficijent uzgona}%\refeq}%
\nomenclature[ecax]{$c$}{duljina tetive, [m]}%

\subsection{Primjer potpotpoglavlja}

Slijedi prvi primjer slike: FSB.(pogl.sliku~\ref{fig1})
\nomenclature[Kf]{$FSB$}{Fakultet strojarstva i brodogradnje}%

\begin{figure}[h]
  \centering
  \includegraphics[height=1.2cm]{fsb_logo_n}\\
  \hangcaption{Primjer slike: kod ove slike primjenjeno je
  \emph{vise\'{c}e} zaglavlje -- hangcaption}
  \label{fig1}
\end{figure}

\clearpage
Slijedi mali primjer slike: SUZ.(pogledaj sliku~\ref{fig2})
\nomenclature[Ks]{$SUZ$}{Sveučilište u Zagrebu}%
%prvim slovom unutar uglate zagrade 'K' definiram da li je kratica ili indeks ili
%akcent, a drugim slovom 's' mogu regulirati poredak pojavljivanja u popisu
%oznaka (po abecedi, naravno)

\begin{figure}
  \centering
  \includegraphics[height=3.2cm]{unizg_plavi_t2}\\
  \hangcaption{Primjer slike: kod ove slike primjenjeno je
  \emph{vise\'{c}e} zaglavlje -- hangcaption}
  \label{fig2}
\end{figure}


Pored slike dan je i primjer tablice.

\begin{table}[!h]
\hangcaption{Primjer tablice}
\label{tablica}
  \centering
\begin{tabular}{|c c|} \hline
    A & 1 \\
    B & 2 \\
    C & 3 \\
    D & 4 \\ \hline
\end{tabular}
\end{table}

